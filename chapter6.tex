\documentclass[syuuron]{kuee}
\usepackage[dvipdfmx]{graphicx}
\usepackage{kueecite}

\title{評価構造における単語間の関係性可視化に関する研究}
\author{小澤 啓太}
\professor{小山田 耕二 教授}
\course{京都大学大学院 工学研究科}
\department{電気工学専攻}
\date{平成28年2月4日}

%%% 本文
\begin{document}
\maketitle
\tableofcontents


%%実験結果
\chapter{実験結果}
	\section{比較実験}
	%%%ノード数まとめた 平均時間正答率 棒グラフ
	3種類の可視化手法によるタスクの平均所要時間、平均正答率の結果は、それぞれ図?~?のようになった。
	タスク1の平均所要時間は、ノード数が15個の場合では差が少ないが、ノード数が136個の場合ではネットワーク表示はWord Cloud表示の約3倍以上の時間を要した。
	タスク2の平均所要時間は、ノード数に関係なく提案手法によるWord Cloud表示が他の手法に比べ約半分の時間だった。
	タスク3の平均所要時間は、ノード数に関係なくネットワーク表示はWord Cloud表示の約2倍以上の時間を要した。
	
	次に、分散分析を行うことで、評価項目数や可視化手法によってタスクの時間や正答率に有意差が出るかを検証した。
	分散分析とは、統計的仮説検定の手法の一つであり、
	観測データにおける変動を誤差変動と各要因およびそれらの交互作用による変動に分解することによって、
	要因および交互作用の効果を判定する。
	はじめにタスク1に分散分析を行い、可視化手法によるタスク時間の有意差を検定したところ、ノード数が136個の場合にP値が0.005と棄却域を下回っており、
	有意差があると判定された。
	その後、それぞれの可視化手法間での有意差をt検定を用いて検定したところ、ネットワーク表示と提案手法によるWord Cloud表示ではP値が0.012と棄却域を下回っており、
	有意差があると判定された。
	次に、タスク2に分散分析を行い可視化手法によるタスク時間の有意差を検定したところ、全てのノード数の場合にP値が棄却域を上回っており、
	有意差は確認できなかった。
	しかし、それぞれの可視化手法間での有意差をt検定を用いて検定したところ、
	提案手法によるWord Cloud表示と単語をランダム配置するWord Cloud表示ではP値が0.004と棄却域を下回っており、
	有意差があると判定された。
	最後に、タスク3に分散分析を行い可視化手法によるタスク時間の有意差を検定したところ、
	ノード数が15個、45個の場合ではP値が0.003、0.032と有意差が確認できたが、139個の場合では確認できなかった。
	
	タスクの平均正答率については、図?のようにノード数、可視化手法によってあまり差は見られず、
	有意差の検定のため分散分析、t検定を行ったが有意差は確認できなかった。
		
	\section{ケーススタディ}
	評価構造の分析は、評価構造作成の際の被験者によって行われた。
	分析の対象である評価構造は図?に表す。
	はじめに、評価構造に含まれる単語を概観するためにWord Cloudウィンドウを確認した。
	フォントサイズの大きい単語から注目したい単語を探した。
	そこから分析者は「癒す」、「疲れ」という慰安目的の旅行や「美味しい」、「料理」、「食べ物」など食べあるき旅行、
	「運動」、「ストレス」、「発散」など体を動かす旅行などを求めている、
	また、これらの単語の中で「癒す」のフォントサイズが一番大きいことから慰安目的の旅行を求めている人が多いのではないかという仮説を立てた。
	
	次に、仮説を確認するため上記の単語の詳細な情報を探索した。
	分析者は「癒す」をクリックし、どのような評価項目で使われているかを評価項目リストウィンドウから確認した。
	「癒す」は「マイナスイオンで癒やされたい」、「癒される」、「疲れを癒やすことができる」という評価項目で使用されていた。
	その中で、「マイナスイオンで癒やされたい」の詳細について分析者たちは興味を持った。
	
	次に上位、下位の評価項目を探索した。
	分析者は評価項目リストウィンドウの「マイナスイオンで癒やされたい」以外のチェックボックスを外し、
	ネットワーク図内で評価項目の絞り込みを行った。
	「マイナスイオンで癒やされたい」の評価項目は紫色、上位に接続された辺と評価項目は青色、
	下位に接続された辺と評価項目は赤色で強調表示されている。
	ネットワークウィンドウからは「マイナスイオンで癒やされたい」ことの好ましい理由として「嬉しい」という上位項目が得られた。
	同様に、「マイナスイオンで癒やされたい」ための条件として「自然で尽きられた場所」、「海」「山」という下位項目が得られた。
	
	このケーススタディによって、提案システムでは対話的な操作によって評価構造の効果的な分析が可能であることが示された。
	提案システムでは、従来のネットワーク図のみでは発見が困難であった多くの人が共有する頻出語、頻出語と因果関係を持つ単語の発見を促進し、
	評価構造の概観の効率化を行った。
	
	\section{ユーザーフィードバック}
		\subsection{5段階評価アンケート結果}
		比較実験後のアンケート結果は図?~?のようになった。
		「新しい発見や仮説構築に繋がる示唆は得られそうか」についての評価はネットワーク表示、提案手法によるWord Cloud表示、単語をランダム配置するWord Cloud表示の順になった。
		可視化手法を要因とした分散分析を行うと、P値は0.230となり、棄却域の0.05を上回ることから可視化手法の水準間に有意差がないことがわかった。
		次に、「多くの人が共有する評価項目を発見することに有効であるか」についての評価は提案手法によるWord Cloud表示、単語をランダム配置するWord Cloud表示、ネットワーク表示の順になった。
		可視化手法を要因とした分散分析を行うと、P値は0.003となり、棄却域を下回ることから可視化手法の水準間に有意差があることがわかった。
		2つの可視化手法間の有意差を検定するためにt検定を行ったところ、提案手法によるWord Cloud表示とネットワーク表示ではP値が0.004となり、
		棄却域を下回ったのでWord Cloud表示の場合、多くの人が共有する評価項目の発見に優れることが示された。
		最後に、「評価項目間の因果関係を発見することに有効であるか」についての評価はネットワーク表示、提案手法によるWord Cloud表示、単語をランダム配置するWord Cloud表示の順になった。
		可視化手法を要因とした分散分析を行うと、P値は0.0504となり、棄却域を上回ることから可視化手法の水準間に有意差がないことがわかった。
		しかし、2つの可視化手法間の有意差を検定するためにt検定を行ったところ、提案手法によるWord Cloud表示と単語をランダム配置するWord Cloud表示ではP値が0.040となり、
		棄却域を下回ったのでWord Cloud表示の場合、提案手法を用いるほうが評価項目間の因果関係を発見することに有効であことが示された。
	
		\subsection{自由記述アンケート結果}
		自由記述アンケート結果を以下に示す。
		\begin{itemize}
			\item ネットワーク表示では拡大と移動を行わなければ文章を読み取ることができないが、Word Cloud表示ではそのような操作を必要とせず文章を概観することができる。
			\item 提案手法によるWord Cloud表示では、単語をクリックし関連する単語が強調表示される際に、ランダム配置と比較して、クリックした単語の近くに配置されていることが多いので単語間の関係性が読み取りやすいという点で優れている。
			\item 提案手法によるWord Cloud表示では、ランダム配置と比較して空白部分が多く、指定された単語を図中から探すことがやや難しかった。
			\item Word Cloud表示では、同じ大きさの単語が横に並べて配置された時に単語の切れ目を認識することが難しかった。
			\item 提案手法によるWord Cloud表示のほうが単語同士の距離が広くとられていて目的のノードを見つけやすかった。
			\item Word Cloud表示では、色と線が表示されるのがいい。ネットワーク図との連携があれば更に上位(下位)の項目をたどることが可能になると考えられる。
			\item ある特定の単語にマウスオーバーすることで、関連する単語が浮かび上がる動作は、明確な検索ワードを持ち合わせていないけれども探索を行いたい時にあると便利だと感じました。
			\item 矢印がなくても、言葉の間のある程度の関係性が予想でき、そのあとで矢印で確認することができるため、より自由な発想につながる可能性を感じました。
			\item 矢印は見づらかった。色で上位下位関係は分かるので矢印ではなく線でいいのではないかと感じました。
			\item 空白部分があるので全体的に文字サイズを少し大きく出来るのではないかと感じました。
			\item テキスト入力した項目がクリックされる機能もほしい。
		\end{itemize}
	


%======================================================================
%		謝辞
%======================================================================
\begin{acknowledgements}
	ほげ
\end{acknowledgements}



%======================================================================
%		参考文献
%======================================================================
\bibliographystyle{kueethesis}
\bibliography{sotsuron}
\begin{thebibliography}{数字}
	\bibitem{egm1} 奥西智哉, 炊飯米を生地に添加したパンの官能評価. 日本食品科学工学会誌, 56, 424-428, (2009).
	\bibitem{egm2} 入江正和, 豚肉質の評価法. 日本養豚学会誌, 39, 221-254, (2002).
	\bibitem{egm3} 来田宣幸, 赤井聡文. 野球における球速と球速感の関係. 日本認知心理学会発表論文集, 42-42, (2009).
	\bibitem{egm4} 中前光弘, 順位法を用いた視覚評価の信頼性について: 順序尺度の解析と正規化順位法による尺度構成法. 日放技学誌, 56, 725-730, (2000).
	\bibitem{egm5} 大山正, 瀧本誓, 岩澤秀紀. 順位法を用いた視覚評価の信頼性について: 順序尺度の解析と正規化順位法による尺度構成法. 行動計量学, 20, 55-64, (1993).
	\bibitem{egm6} J. Sanui, Visualization of users’requirements: Introduction of Evaluation Grid Method, Proceedings of the 3rd Design and Decision Support System in Architecture and Urban Planning Conference, 365-374, (1996).
	\bibitem{egm7} 讃井純一郎, 乾正雄. レパートリー・グリッド発展手法による住環境評価構造の抽出:認知心理学に基づく住環境評価に関する研究(1). 日本建築学会計画系論文報告集, 367, 15-22, (1986).
	\bibitem{egm8} 尾上洋介, 久木元伸如, 小山田耕二. 可視化情報学会における会員満足度の因果関係分析. 可視化情報学会論文集, 34, 43-51, (2014).
	\bibitem{egm9} 本村陽一, 金出武雄. ヒトの認知・評価構造の定量化モデリングと確率推論. 電子情報通信学会技術研究報告, 104, 25-30, (2005).
	\bibitem{rg1} G. A. Kelly, The Psychology of Personal Constructs, 1 and 2, (1955).
	\bibitem{net1} Y. Onoue, N. Kukimoto, N. Sakamoto, K. Koyamada, Network Coarse-Graining for Evaluation Structures, In Proc. of International Conference on Simulation Technology, 34, 447-450, (2015). 
	\bibitem{kh1} 樋口耕一, テキスト型データの計量的分析. 理論と方法, 19, 101-115, (2004).
	\bibitem{wg1} Riehmann. P, Gruendl. H, Potthast. M, Trenkmann. M, Stein. B, Froehlich. B, WORDGRAPH: Keyword-in-Context Visualization for NETSPEAK's Wildcard Search. Visualization and Computer Graphics, IEEE Transactions on 18.9, 1411-1423, (2012).
	\bibitem{rwc1} Strobelt. H, Spicker. M, Stoffel. A, Keim. D, Deussen. O, Rolled‐out Wordles: A Heuristic Method for Overlap Removal of 2D Data Representatives, Computer Graphics Forum, 31, 1135-1144, (2012).
	\bibitem{fta1} Huang. X, Lai. W, Force-transfer: a new approach to removing overlapping nodes in graph layout, Proceedings of the 26th Australasian computer science conference, 16, 349-358, (2003).
	\bibitem{or1} Gomez-Nieto. E, San Roman. F, Pagliosa. P, Casaca. W, Helou. E. S, Oliveira. M. C. F, Nonato. L. G, Similarity Preserving Snippet-Based Visualization of Web Search Results, Visualization and Computer Graphics, IEEE Transactions on, 20, 457-470, (2014).
	\bibitem{or2} Gomez-Nieto. E, Casaca. W, Motta. D, Hartmann. I, Taubin. G, Nonato. L, Dealing with Multiple Requirements in Geometric Arrangements, Visualization and Computer Graphics, IEEE Transactions on, 1, (2015).
	\bibitem{mcb1} T. Kuo, K. Yamamoto, Y. Matsumoto, Applying Conditional Random Fields to Japanese Morphological Analysis, Proceedings of the 2004 conference on empirical methods in natural language processing, 230-237, (2004).
	\bibitem{hak1} 尾上洋介, 評価構造のビジュアル分析に関する研究,博士論文, 2016.
\end{thebibliography}

%======================================================================
%		付録
%======================================================================
\appendix


\end{document}
% Local Variables:
% fill-column: 70
% End:
