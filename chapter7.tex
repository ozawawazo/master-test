\documentclass[syuuron]{kuee}
\usepackage[dvipdfmx]{graphicx}
\usepackage{kueecite}

\title{評価構造における単語間の関係性可視化に関する研究}
\author{小澤 啓太}
\professor{小山田 耕二 教授}
\course{京都大学大学院 工学研究科}
\department{電気工学専攻}
\date{平成28年2月4日}

%%% 本文
\begin{document}
\maketitle
\tableofcontents

			
%%考察
\chapter{考察}
	\section{比較実験}
	ここでは比較実験についての考察を行う。
	タスク1の結果について、ネットワーク表示ではノード数が増えるにつれタスク所要時間が増えていき、序論で取り上げた問題と同じ結果が起きた。
	その一方で、提案手法によるWord Cloud表示では、ノード数が増えてもタスク所要時間は変動しなかった。
	また、正答率に関して、提案手法によるWord Cloud表示はネットワーク表示では有意差はなく、誤った情報を与える表示では無いことがわかった。
	この結果から、提案手法によるではWord Cloud表示ではネットワーク表示による問題を解決していると考えられる。
	
	タスク2の結果から、提案手法による単語の配置計算の優位性を確かめることができた。
	この差の原因は、隣接する評価項目の単語が近くに配置されているためであると考えられる。
	また、ノード数が増えるにつれ、タスク所要時間の差が増加することから、大規模な評価構造分析の場合でも提案手法によるではWord Cloud表示が有効であると考えられる。
	また、ネットワーク表示ではタスク1とで同様に、ノード数が増えるにつれタスク所要時間が増えていき、序論で取り上げた問題と同じ結果が起きた。
	しかし、ネットワーク図可視化と提案手法によるWord Cloud表示間では所要時間での有意差はなかった。
	提案手法によるWord Cloud表示ではノード数にかかわらず約半分の時間だったにも関わらず有意差がなかった原因は、
	評価構造への習熟度に応じて所要時間が大きく異なったことが原因と考えられた。
	そこで、タスク所要時間の順位に関して有意差を検定したところ、ノード数が136個の場合で有意差を確認することが出来た。
	ここから、提案手法によるではWord Cloud表示ではネットワーク表示による問題を解決していると考えられる。
	
	タスク3の結果は、タスク1と同様にネットワーク表示ではノード数が増えるにつれタスク所要時間が増えていき、序論で取り上げた問題と同じ結果が起きた。
	その一方で、提案手法によるWord Cloud表示では、ノード数が増えてもタスク所要時間の増加量は少なかった。
	また、正答率に関して、提案手法によるWord Cloud表示はネットワーク表示では有意差はなく、誤った情報を与える表示では無いことがわかった。
	この結果から、提案手法によるではWord Cloud表示ではネットワーク表示による問題を解決していると考えられる。
	
	\section{ケーススタディ}%%%尾上さん論文5.5.1が参考になるかも
	提案手法を3人の被験者に使用してもらい得た評価について考察する。
		\paragraph{有効性}
		有効性に関して4種類の観点から評価された。
		一つ目は評価構造の概観に関してである。
		ワードクラウドウィンドウは評価構造の概観に有効であり、評価構造内にどのような評価項目があるのか予想がつきやすくなった。
		二つ目は重要項目の発見に関してである。
		ワードクラウドウィンドウのフォントサイズによって評価構造の中で頻出する項目を発見でき、
		重要な項目の発見につながった。
		三つ目は評価項目間の関係性を把握に関してである。
		ワードクラウドウィンドウ内の注目単語をクリックすることで、クリックした単語と隣接する評価項目の単語を確認することができた。
		また、クリックした単語が使用されている評価項目が評価構造ウィンドウ内で強調されることで隣接する評価項目だけでなく、
		上位、下位概念の項目を全て確認することができた。
		四つ目は評価項目のグループ分けである。
		ワードクラウドウィンドウで評価構造内で隣接関係にある単語が近くに配置されているため、グループ分けを行うことができた。
		しかし、全ての単語がグループ分けできている訳ではなく、複数グループの評価項目に出現する単語などはグループ分けできなかった。
		
		\paragraph{操作性}
		操作性に関しては良い評価を得た。
		提案システムは、複数の操作ステップを必要とせず、また直感的な操作であったため、利用者は少しの説明と訓練で提案システムを使えるようになる。
		さらに、Webアプリケーションとして使用できるため、準備にかかる手間も少なく使いやすい。
		
		\paragraph{改善点}
		改善点について、評価項目と回答者の関係可視化についての意見を得た。
		ワードクラウドウィンドウで大きく表示されている単語は多くの回答者が注目している項目だと分かるものの、
		回答者数や回答者名の表示はなく、より詳細な分析を行うためには回答者の情報を表示することが必要である。
		二つ目は、表示方法についてである。
		
		以上、ケーススタディから得た評価の多くは肯定的であり、提案システムの有用性は確認できた。
		提案システムはシステム開発をする前に得たシステム要件を4つ満たしており目的を達成できていること言える。
		しかし、改善すべき点もあり今後の課題になる。
		
	\section{ユーザーフィードバック}
		\subsection{5段階評価アンケート結果}
		この項では、5段階評価アンケート結果についての考察を行う。
		提案手法によるWord Cloud表示では、ネットワーク表示の問題点である多くの人が共有する評価項目の発見に関して
		有効でありつつ、評価項目の関係性を同時に可視化している手法であることが確認できた。
		しかし、新しい発見や仮説構築につなげることへの有効性はネットワーク図と差は確認できなかった。
		この原因は、2つの可視化手法の可視化内容の違いだと考えられる。
		提案手法によるWord Cloud表示では、多くの人が共有する評価項目と評価項目の関係性を同時に可視化しているが、
		評価項目の関係性については詳細な情報を伝えることはできない。
		注目する単語の評価項目と隣接する評価項目の詳細な情報や、より上位、下位の評価項目の情報を確認することはできない。
		そこで、今後は隣接する評価項目の詳細な情報や、より上位、下位の評価項目の情報を可視化する手法や、
		提案システムのようなネットワーク図とWord Cloud表示の同時表示するような機能が必要であると考えられる。
		
		\subsection{自由記述アンケート結果}
		この項では、自由記述アンケート結果に対する考察を行う。
		アンケートの結果は大きく2種類に分けることができた。
		一つ目は、提案手法であるWord Cloudの単語の配置計算についてである。
		単語の距離関係から単語間の関係性の予測が可能になり、その予測の確認を単語の文字色や矢印で行うことができる。
		このような単語の関係性を直感的に予測でき、その予測の正否の確認を即座に行える機能が有効であると確認できた。
		その一方で、提案手法によるWord Cloud可視化では、空白部分が目立ち単語の探索に悪影響をきたすことがあった。
		単語の文字サイズを拡大することなどでの空白部分の削減は今後の課題とされる。
		また、単語を配置する際にフォントサイズが同じ複数の単語が横に並べられることで、
		複数の単語を一塊の単語と勘違いすることも確認された。単語間の距離感の設定も今後の課題とされる。
	
		二つ目は、インタラクションについてである。
		単語をクリックした際に発生する操作方法について、よりシンプルなデザインにすることや
		単語の探索機能、より分かりやすい関連語の表示など新しい機能を採用することで改善することが求められている。
	

%======================================================================
%		謝辞
%======================================================================
\begin{acknowledgements}
	ほげ
\end{acknowledgements}



%======================================================================
%		参考文献
%======================================================================
\bibliographystyle{kueethesis}
\bibliography{sotsuron}
\begin{thebibliography}{数字}
	\bibitem{egm1} 奥西智哉, 炊飯米を生地に添加したパンの官能評価. 日本食品科学工学会誌, 56, 424-428, (2009).
	\bibitem{egm2} 入江正和, 豚肉質の評価法. 日本養豚学会誌, 39, 221-254, (2002).
	\bibitem{egm3} 来田宣幸, 赤井聡文. 野球における球速と球速感の関係. 日本認知心理学会発表論文集, 42-42, (2009).
	\bibitem{egm4} 中前光弘, 順位法を用いた視覚評価の信頼性について: 順序尺度の解析と正規化順位法による尺度構成法. 日放技学誌, 56, 725-730, (2000).
	\bibitem{egm5} 大山正, 瀧本誓, 岩澤秀紀. 順位法を用いた視覚評価の信頼性について: 順序尺度の解析と正規化順位法による尺度構成法. 行動計量学, 20, 55-64, (1993).
	\bibitem{egm6} J. Sanui, Visualization of users’requirements: Introduction of Evaluation Grid Method, Proceedings of the 3rd Design and Decision Support System in Architecture and Urban Planning Conference, 365-374, (1996).
	\bibitem{egm7} 讃井純一郎, 乾正雄. レパートリー・グリッド発展手法による住環境評価構造の抽出:認知心理学に基づく住環境評価に関する研究(1). 日本建築学会計画系論文報告集, 367, 15-22, (1986).
	\bibitem{egm8} 尾上洋介, 久木元伸如, 小山田耕二. 可視化情報学会における会員満足度の因果関係分析. 可視化情報学会論文集, 34, 43-51, (2014).
	\bibitem{egm9} 本村陽一, 金出武雄. ヒトの認知・評価構造の定量化モデリングと確率推論. 電子情報通信学会技術研究報告, 104, 25-30, (2005).
	\bibitem{rg1} G. A. Kelly, The Psychology of Personal Constructs, 1 and 2, (1955).
	\bibitem{net1} Y. Onoue, N. Kukimoto, N. Sakamoto, K. Koyamada, Network Coarse-Graining for Evaluation Structures, In Proc. of International Conference on Simulation Technology, 34, 447-450, (2015). 
	\bibitem{kh1} 樋口耕一, テキスト型データの計量的分析. 理論と方法, 19, 101-115, (2004).
	\bibitem{wg1} Riehmann. P, Gruendl. H, Potthast. M, Trenkmann. M, Stein. B, Froehlich. B, WORDGRAPH: Keyword-in-Context Visualization for NETSPEAK's Wildcard Search. Visualization and Computer Graphics, IEEE Transactions on 18.9, 1411-1423, (2012).
	\bibitem{rwc1} Strobelt. H, Spicker. M, Stoffel. A, Keim. D, Deussen. O, Rolled‐out Wordles: A Heuristic Method for Overlap Removal of 2D Data Representatives, Computer Graphics Forum, 31, 1135-1144, (2012).
	\bibitem{fta1} Huang. X, Lai. W, Force-transfer: a new approach to removing overlapping nodes in graph layout, Proceedings of the 26th Australasian computer science conference, 16, 349-358, (2003).
	\bibitem{or1} Gomez-Nieto. E, San Roman. F, Pagliosa. P, Casaca. W, Helou. E. S, Oliveira. M. C. F, Nonato. L. G, Similarity Preserving Snippet-Based Visualization of Web Search Results, Visualization and Computer Graphics, IEEE Transactions on, 20, 457-470, (2014).
	\bibitem{or2} Gomez-Nieto. E, Casaca. W, Motta. D, Hartmann. I, Taubin. G, Nonato. L, Dealing with Multiple Requirements in Geometric Arrangements, Visualization and Computer Graphics, IEEE Transactions on, 1, (2015).
	\bibitem{mcb1} T. Kuo, K. Yamamoto, Y. Matsumoto, Applying Conditional Random Fields to Japanese Morphological Analysis, Proceedings of the 2004 conference on empirical methods in natural language processing, 230-237, (2004).
	\bibitem{hak1} 尾上洋介, 評価構造のビジュアル分析に関する研究,博士論文, 2016.
\end{thebibliography}

%======================================================================
%		付録
%======================================================================
\appendix


\end{document}
% Local Variables:
% fill-column: 70
% End:
