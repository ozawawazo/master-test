\documentclass[syuuron]{kuee}
\usepackage[dvipdfmx]{graphicx}
\usepackage{kueecite}

\title{評価構造における単語間の関係性可視化に関する研究}
\author{小澤 啓太}
\professor{小山田 耕二 教授}
\course{京都大学大学院 工学研究科}
\department{電気工学専攻}
\date{平成28年2月4日}

%%% 本文
\begin{document}
\maketitle
\tableofcontents


%%評価実験
\chapter{評価実験}
	本章では, 提案手法、提案システムの有効性を証明するために行った2組の比較実験とケーススタディ, ユーザーフィードバックの評価実験について述べる. 
	\section{比較実験}
		\subsection{比較対象}
			実験では, 2組の比較実験を行った. 
			1組目は提案手法による可視化結果と従来のネットワーク表示の比較である. 
			この比較実験により, Word Cloudによる可視化の有効性を証明する. 
			2組目は提案手法のようにWord Cloudの単語の座標を計算して配置した可視化結果と, 配置座標をランダムに決定した可視化結果との比較結果を行った. 
			こちらの比較実験ではWord Cloudの配置座標を考慮することの有効性を証明する. 
			比較実験で使用する評価構造ネットワーク、Word Cloudのインタラクションはそれぞれ前章で説明した内容の中で、
			ネットワークとWord Cloud間の連携部分以外は搭載されている。
			また、Word Cloudには評価項目リスト画面が搭載されている。
			
		\subsection{実験データ}
			実験には3種類の評価構造ネットワークデータを使用する. 
			評価構造のデータとして, E-gridを使用したインタビュー結果から得られたデータを使用した.
			インタビューは日本語で、回答者と質問者の1対1で行われた。
			評価項目数はそれぞれ15個, 45個, 136個の3種類用意した. 
			評価項目数が15個のデータは欲しい自動車についての評価構造である。
			このデータは1人の回答者の評価構造であり、形態素解析によって得た37個の形態素をWord Cloudで表示した。
			評価項目数が45個のデータは行きたい旅行先についての評価構造である。
			このデータは3人の回答者の評価構造を統合した全体評価構造であり、形態素解析によって得た70個の形態素をWord Cloudで表示した。
			評価項目数が136個のデータは欲しいシャープペンシルについての評価構造である。
			このデータは16人の回答者の評価構造を統合した全体評価構造であり、形態素解析によって得た172個の形態素をWord Cloudで表示した。
		
		\subsection{実験内容}
			3種類の評価構造データの可視化に対してそれぞれ3種類のタスクを用意した. 
			実験内容はシステム要件で述べられた以下の三つの観点を考慮して作成した。
			\begin{enumerate}
				\item 評価構造の全体把握はできるレイアウト表示になっているか
				\item 評価項目、単語の上位項目と下位項目をわかりやすく可視化しているか
				\item 重要な評価項目をわかりやすく表示にしているか
			\end{enumerate}

			システム要件1では、観点1のように評価構造の要約や予想外の要素の特定など評価構造の全体を把握することが望まれていた。
			そこでタスク1では、評価構造内に存在する評価項目をランダムに指定し、発見するというタスクを設定した。
			頻出する評価項目だけでなく、1度しか回答されていない評価項目など様々な評価項目を発見する時間、精度を計測することで
			システム要件(1)への有効度を確かめた。
			システム要件(2)では、観点2のように評価項目の上位項目と下位項目の発見が望まれていた。
			そこでタスク2では、評価構造内に存在する評価項目をランダムに指定し、その項目と隣接する項目を発見するというタスクを設定した。
			評価項目の上位、下位項目の発見の時間、精度を計測することで観点(2)への有効度を確かめた。
			システム要件(3)では、観点3のように複数の実感参加者から回答される評価項目の特定が望まれていた。
			そこでタスク3では、評価構造内で複数の実感参加者から回答された評価項目を発見するというタスクを設定した。
			タスク3を行うことでシステム要件(3)への有効度を確かめた。
			
			実験用のプログラムはWebベースで開発を行い, 結果の記録や問題の生成を自動化した. 
			評価指標としてタスクを完了するまでの時間、タスク結果の正否を記録した。
		
		\subsection{実験手順}
			実験参加者は10人である. 
			はじめに評価構造のネットワーク表示とWord Cloud表示の特徴を説明し、
			ネットワークとWord Cloud可視化結果のサンプルシステムを使用させシステムの使用法を学ばせた。
			説明の後にタスクの内容説明と例を示し、3種類の評価構造を3種類で可視化したタスク計9個を行わせた。
			実験参加者毎に選択肢を生成し, 順番に実行するように指示をした. 
			実験には, ネットワーク図が画面内に収まる十分な大きさのディスプレイと. 使い慣れたポインティングデバイスを使用するように指示をした. 
		
	\section{ケーススタディ}
		本節では, 提案システムの使用例を通じた評価を行う。
		提案システムを用いたケーススタディを1名が行い, 提案システムで評価構造の頻出語や頻出語から帰結する評価基準の探索を行った. 
		実験に使用した機材はCPU がIntel Core i3,メモリが4.00GB,OSがwindows 8.1、ディスプレイは10.6インチ液晶ディスプレイ、解像度1920×1080のPC である.
		提案システムはGoogle Chromeから使用した。
		参加者にはシステムの使用方法について事前にサンプルデータを用いたシステムを見せて説明した. 以下では実験に使用したデータについて述べる. 
		
		このケーススタディでは, 「行きたい旅行先」に関する調査で得られた全体評価構造を使用した. 
		全体評価構造は7人の被験者による評価構造から構成される。
		評価構造はE-gridを使用したインタビューによって作成された. 
		インタビューは日本語で、被験者と質問者の1対1で行われた。
		刺激要素として、被験者が行きたい旅行先をそれぞれ4つ使用した。
		インタビューの結果89個の評価項目と103個の評価項目間の接続による全体評価構造が得られた。
		被験者らは, 旅行計画を考えており, その際に多くの旅行参加者の希望をかなえた旅行計画を作成するために提案システムを使用した. 
		Word Cloudでは評価構造のテキストデータの中から名詞, 動詞, 形容詞を抽出して可視化した. 
		また, 名詞のうち数字は除外した. 
		ケーススタディ参加者には,実験後に提案システムの有効な点や改善が必要な点、追加すべき機能など自由記述アンケートを行った.
		
	\section{ユーザーフィードバック}
		比較実験後に10名の参加者にはシステムに関するアンケートを行った. 
		アンケートでは3つの可視化結果について各項目5 段階評価させた。
		アンケートの項目は以下である. 
		\begin{itemize}
			\item 新しい発見や仮説構築につながる示唆が得られそうか
			\item 多くの人が共有する評価項目を発見することに有効であるか
			\item 評価項目間の因果関係を発見することに有効であるか
		\end{itemize}


%======================================================================
%		参考文献
%======================================================================
\bibliographystyle{kueethesis}
\bibliography{sotsuron}
\begin{thebibliography}{数字}
	\bibitem{egm1} 奥西智哉, 炊飯米を生地に添加したパンの官能評価. 日本食品科学工学会誌, 56, 424-428, (2009).
	\bibitem{egm2} 入江正和, 豚肉質の評価法. 日本養豚学会誌, 39, 221-254, (2002).
	\bibitem{egm3} 来田宣幸, 赤井聡文. 野球における球速と球速感の関係. 日本認知心理学会発表論文集, 42-42, (2009).
	\bibitem{egm4} 中前光弘, 順位法を用いた視覚評価の信頼性について: 順序尺度の解析と正規化順位法による尺度構成法. 日放技学誌, 56, 725-730, (2000).
	\bibitem{egm5} 大山正, 瀧本誓, 岩澤秀紀. 順位法を用いた視覚評価の信頼性について: 順序尺度の解析と正規化順位法による尺度構成法. 行動計量学, 20, 55-64, (1993).
	\bibitem{egm6} J. Sanui, Visualization of users’requirements: Introduction of Evaluation Grid Method, Proceedings of the 3rd Design and Decision Support System in Architecture and Urban Planning Conference, 365-374, (1996).
	\bibitem{egm7} 讃井純一郎, 乾正雄. レパートリー・グリッド発展手法による住環境評価構造の抽出:認知心理学に基づく住環境評価に関する研究(1). 日本建築学会計画系論文報告集, 367, 15-22, (1986).
	\bibitem{egm8} 尾上洋介, 久木元伸如, 小山田耕二. 可視化情報学会における会員満足度の因果関係分析. 可視化情報学会論文集, 34, 43-51, (2014).
	\bibitem{egm9} 本村陽一, 金出武雄. ヒトの認知・評価構造の定量化モデリングと確率推論. 電子情報通信学会技術研究報告, 104, 25-30, (2005).
	\bibitem{rg1} G. A. Kelly, The Psychology of Personal Constructs, 1 and 2, (1955).
	\bibitem{net1} Y. Onoue, N. Kukimoto, N. Sakamoto, K. Koyamada, Network Coarse-Graining for Evaluation Structures, In Proc. of International Conference on Simulation Technology, 34, 447-450, (2015). 
	\bibitem{kh1} 樋口耕一, テキスト型データの計量的分析. 理論と方法, 19, 101-115, (2004).
	\bibitem{wg1} Riehmann. P, Gruendl. H, Potthast. M, Trenkmann. M, Stein. B, Froehlich. B, WORDGRAPH: Keyword-in-Context Visualization for NETSPEAK's Wildcard Search. Visualization and Computer Graphics, IEEE Transactions on 18.9, 1411-1423, (2012).
	\bibitem{rwc1} Strobelt. H, Spicker. M, Stoffel. A, Keim. D, Deussen. O, Rolled‐out Wordles: A Heuristic Method for Overlap Removal of 2D Data Representatives, Computer Graphics Forum, 31, 1135-1144, (2012).
	\bibitem{fta1} Huang. X, Lai. W, Force-transfer: a new approach to removing overlapping nodes in graph layout, Proceedings of the 26th Australasian computer science conference, 16, 349-358, (2003).
	\bibitem{or1} Gomez-Nieto. E, San Roman. F, Pagliosa. P, Casaca. W, Helou. E. S, Oliveira. M. C. F, Nonato. L. G, Similarity Preserving Snippet-Based Visualization of Web Search Results, Visualization and Computer Graphics, IEEE Transactions on, 20, 457-470, (2014).
	\bibitem{or2} Gomez-Nieto. E, Casaca. W, Motta. D, Hartmann. I, Taubin. G, Nonato. L, Dealing with Multiple Requirements in Geometric Arrangements, Visualization and Computer Graphics, IEEE Transactions on, 1, (2015).
	\bibitem{mcb1} T. Kuo, K. Yamamoto, Y. Matsumoto, Applying Conditional Random Fields to Japanese Morphological Analysis, Proceedings of the 2004 conference on empirical methods in natural language processing, 230-237, (2004).
	\bibitem{hak1} 尾上洋介, 評価構造のビジュアル分析に関する研究,博士論文, 2016.
\end{thebibliography}

%======================================================================
%		付録
%======================================================================
\appendix


\end{document}
% Local Variables:
% fill-column: 70
% End:
