\documentclass[syuuron]{kuee}
\usepackage[dvipdfmx]{graphicx}
\usepackage{kueecite}

\title{評価構造における単語間の関係性可視化に関する研究}
\author{小澤 啓太}
\professor{小山田 耕二 教授}
\course{京都大学大学院 工学研究科}
\department{電気工学専攻}
\date{平成28年2月4日}

%%% 本文
\begin{document}
\maketitle
\tableofcontents


%%%結論と課題
\chapter{結論と課題}
	\section{結論}
		本論文では、評価構造の大規模化に伴う見やすさの低下、それによる評価構造分析の非効率化という問題に対して、
		評価項目内に出現する単語の頻度と,評価構造内での単語間の位置関係を反映した評価構造データ向けのテキストベースの可視化手法を提案し, 
		提案手法を用いた評価構造分析システムの開発を行った. 
		評価実験によって確認した提案手法の有効性は次の3点である。
		
		はじめに、評価構造内の重要な評価項目の発見の効率化について述べる。
		比較実験では、評価構造内で頻出する単語を発見するタスクの所要時間が従来のネットワーク表示よりも短く、
		5段階評価アンケートからも重要な評価項目の発見に関して従来のネットワーク表示よりも有効であることが確認できた。
		考察でも述べたように、提案手法では頻出する単語の文字サイズを大きくし、また空白部分を出来るだけ減らしその分文字サイズを大きく表示させた。
		これにより重要な単語、評価項目の発見を容易にした。
		
		次に、評価構造内での単語の関係性の可視化について述べる。
		比較実験から、選択された単語と隣接する単語の発見するタスクの所要時間に関して、従来のネットワーク表示よりも短く、
		単語をランダムに配置したWord Cloud表示と比較しても短く、提案手法の有効性を確かめることができた。
		
		次に、評価構造の概観に関して述べる。
		比較実験から、ノード数が増えた場合でも単語の発見の時間に変化はなかった。
		またケーススタディからも、評価構造の概観が可能であることが確認できた。
		
		また、ケーススタディから提案システムの有効性を確認することができた。
		提案手法によるWord Cloud表示とネットワーク図の連携、直感的で簡潔な操作などによって
		評価構造の概観から注目語の発見、注目語の詳細情報の確認、グループ分けなどに有効であることが示された。		
		%%これは、従来のネットワーク表示と比較し、エッジや助詞助動詞などの意味無し語を削減し、複数回出現する語を統合することによって、文字サイズを大きく表示することができたのが
		
	\section{課題}
		本論文で得られた結果を踏まえて、今後検討するべき課題を以下で述べる。
		提案手法に関しては3種類あげられる。
		
		一つ目は、単語の配置座標に関してである。
		提案手法による単語の配置座標計算では、隣接関係にある単語が近くに配置することを目的とし、その有効性を確かめることができた。
		今後はより多くの隣接語が近くに配置する計算手法が求められる。
		そのためには新し制約条件を加えた計算や、力学モデルを考慮した座標計算などが考えられる。
		また、提案手法ではその際に単語間の距離が近ずぎた場合にひとまとまりの単語と誤った解釈を与える事があったので、
		このようなことをなくすように制約条件を変更させることも求められる。
		
		二つ目は、単語の文字サイズに関してである。
		文字のサイズを大きくさせることは、評価構造の概観をより容易にするためには求められる。
		提案手法では、空白部分がまだ多く残っていたのでこれをさらに削減させ、その分文字サイズを大きくすることが改善点の一つとしてあげられる。
		また、単語の文字サイズの決定の際に頻出語の線形比ではなく、対数比などにすることで最小単語の文字サイズを大きくすることも検証する価値はある。
		
		三つ目は、より詳細な情報の可視化に関してである。
		ケーススタディで、Word Cloud内の注目単語の回答者人数や回答者名が分析には求められて居ることがわかった。
		このように、より多くの情報を可視化することがより効果的な分析には求められている。
		また、単語をマウスオーバーすることで、その単語が使用されている評価項目を表示する機能など、
		より簡単な操作で情報を表示する機能なども必要である。
		この他にも単語の検索機能や、品詞別で表示単語の絞り込み、副詞など提案手法では表示しなかった品詞の可視化などが
		より効果的な分析に必要だと考えられる。
		
		四つ目は隣接関係の表示である。
		提案手法では、Word Cloud内の注目単語をクリックすることで隣接する単語の表示を行うが、
		より上位、下位の評価項目の表示が求められていることがわかった。
		また、提案手法ではわからなかった注目単語と隣接する評価項目の文章の可視化も求められる。
		
		また、提案システムに関しては、ケーススタディからインタラクションについての改善が求められていることがわかった。
		よりシンプルなデザインやWord Cloudウィンドウのズーム機能、強調表示の改善などウィンドウごとの機能の向上と、
		機能間の連携をより向上させることが評価構造のより効果的な分析に繋がることがわかった。
			
		

		

%======================================================================
%		参考文献
%======================================================================
\bibliographystyle{kueethesis}
\bibliography{sotsuron}
\begin{thebibliography}{数字}
	\bibitem{egm1} 奥西智哉, 炊飯米を生地に添加したパンの官能評価. 日本食品科学工学会誌, 56, 424-428, (2009).
	\bibitem{egm2} 入江正和, 豚肉質の評価法. 日本養豚学会誌, 39, 221-254, (2002).
	\bibitem{egm3} 来田宣幸, 赤井聡文. 野球における球速と球速感の関係. 日本認知心理学会発表論文集, 42-42, (2009).
	\bibitem{egm4} 中前光弘, 順位法を用いた視覚評価の信頼性について: 順序尺度の解析と正規化順位法による尺度構成法. 日放技学誌, 56, 725-730, (2000).
	\bibitem{egm5} 大山正, 瀧本誓, 岩澤秀紀. 順位法を用いた視覚評価の信頼性について: 順序尺度の解析と正規化順位法による尺度構成法. 行動計量学, 20, 55-64, (1993).
	\bibitem{egm6} J. Sanui, Visualization of users’requirements: Introduction of Evaluation Grid Method, Proceedings of the 3rd Design and Decision Support System in Architecture and Urban Planning Conference, 365-374, (1996).
	\bibitem{egm7} 讃井純一郎, 乾正雄. レパートリー・グリッド発展手法による住環境評価構造の抽出:認知心理学に基づく住環境評価に関する研究(1). 日本建築学会計画系論文報告集, 367, 15-22, (1986).
	\bibitem{egm8} 尾上洋介, 久木元伸如, 小山田耕二. 可視化情報学会における会員満足度の因果関係分析. 可視化情報学会論文集, 34, 43-51, (2014).
	\bibitem{egm9} 本村陽一, 金出武雄. ヒトの認知・評価構造の定量化モデリングと確率推論. 電子情報通信学会技術研究報告, 104, 25-30, (2005).
	\bibitem{rg1} G. A. Kelly, The Psychology of Personal Constructs, 1 and 2, (1955).
	\bibitem{net1} Y. Onoue, N. Kukimoto, N. Sakamoto, K. Koyamada, Network Coarse-Graining for Evaluation Structures, In Proc. of International Conference on Simulation Technology, 34, 447-450, (2015). 
	\bibitem{kh1} 樋口耕一, テキスト型データの計量的分析. 理論と方法, 19, 101-115, (2004).
	\bibitem{wg1} Riehmann. P, Gruendl. H, Potthast. M, Trenkmann. M, Stein. B, Froehlich. B, WORDGRAPH: Keyword-in-Context Visualization for NETSPEAK's Wildcard Search. Visualization and Computer Graphics, IEEE Transactions on 18.9, 1411-1423, (2012).
	\bibitem{rwc1} Strobelt. H, Spicker. M, Stoffel. A, Keim. D, Deussen. O, Rolled‐out Wordles: A Heuristic Method for Overlap Removal of 2D Data Representatives, Computer Graphics Forum, 31, 1135-1144, (2012).
	\bibitem{fta1} Huang. X, Lai. W, Force-transfer: a new approach to removing overlapping nodes in graph layout, Proceedings of the 26th Australasian computer science conference, 16, 349-358, (2003).
	\bibitem{or1} Gomez-Nieto. E, San Roman. F, Pagliosa. P, Casaca. W, Helou. E. S, Oliveira. M. C. F, Nonato. L. G, Similarity Preserving Snippet-Based Visualization of Web Search Results, Visualization and Computer Graphics, IEEE Transactions on, 20, 457-470, (2014).
	\bibitem{or2} Gomez-Nieto. E, Casaca. W, Motta. D, Hartmann. I, Taubin. G, Nonato. L, Dealing with Multiple Requirements in Geometric Arrangements, Visualization and Computer Graphics, IEEE Transactions on, 1, (2015).
	\bibitem{mcb1} T. Kuo, K. Yamamoto, Y. Matsumoto, Applying Conditional Random Fields to Japanese Morphological Analysis, Proceedings of the 2004 conference on empirical methods in natural language processing, 230-237, (2004).
	\bibitem{hak1} 尾上洋介, 評価構造のビジュアル分析に関する研究,博士論文, 2016.
\end{thebibliography}

%======================================================================
%		付録
%======================================================================
\appendix


\end{document}
% Local Variables:
% fill-column: 70
% End:
